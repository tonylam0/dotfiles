% PACKAGES & LAYOUT
%==============================================================================
\usepackage[tmargin=2cm,rmargin=1in,lmargin=1in,margin=0.85in,bmargin=2cm,footskip=.2in]{geometry}
\usepackage{amsmath,amsfonts,amsthm,amssymb,mathtools}
\usepackage[varbb]{newpxmath} % Font selection
\usepackage{xfrac}
\usepackage[makeroom]{cancel}
\usepackage{enumitem}
\usepackage{hyperref,theoremref}
\usepackage{bookmark}
\usepackage[most,many,breakable]{tcolorbox}
\usepackage{xcolor}
\usepackage{varwidth}
\usepackage{multicol,array}
\usepackage[ruled,vlined,linesnumbered]{algorithm2e}
\usepackage{import}
\usepackage{xifthen}
\usepackage{pdfpages}
\usepackage{tikz}
\usetikzlibrary{positioning}
\usepackage{titletoc}
\usepackage{forest}

% Basic Colors
\definecolor{doc}{RGB}{0,60,110}
\hypersetup{
    pdftitle={Assignment},
    colorlinks=true, linkcolor=doc!90,
    bookmarksnumbered=true,
    bookmarksopen=true
}

%==============================================================================
% GRAPHICS & FIGURES
%==============================================================================
\newcommand{\incfig}[1]{%
    \def\svgwidth{\columnwidth}
    \import{./figures/}{#1.pdf_tex}
}

\usepackage{tikzsymbols}
\usetikzlibrary{arrows,calc,shadows.blur}

%==============================================================================
% CUSTOM COLORS
%==============================================================================
\definecolor{myg}{RGB}{56, 140, 70}
\definecolor{myb}{RGB}{45, 111, 177}
\definecolor{myr}{RGB}{199, 68, 64}
\definecolor{mytheorembg}{HTML}{F2F2F9}
\definecolor{mytheoremfr}{HTML}{00007B}
\definecolor{mylemmabg}{HTML}{FFFAF8}
\definecolor{mylemmafr}{HTML}{983b0f}
\definecolor{mypropbg}{HTML}{f2fbfc}
\definecolor{mypropfr}{HTML}{191971}
\definecolor{myexamplebg}{HTML}{F2FBF8}
\definecolor{myexamplefr}{HTML}{88D6D1}
\definecolor{myexampleti}{HTML}{2A7F7F}
\definecolor{mydefinitbg}{HTML}{E5E5FF}
\definecolor{mydefinitfr}{HTML}{3F3FA3}
\definecolor{notesgreen}{RGB}{0,162,0}
\definecolor{myp}{RGB}{197, 92, 212}
\definecolor{mygr}{HTML}{2C3338}
\definecolor{myred}{RGB}{127,0,0}
\definecolor{myexercisebg}{HTML}{F2FBF8}
\definecolor{myexercisefg}{HTML}{88D6D1}

%==============================================================================
% TCOLORBOX DEFINITIONS
%==============================================================================

% --- Theorem ---
\newtcbtheorem[number within=chapter]{theorem}{Theorem}{
    enhanced, breakable,
    colback = mytheorembg, frame hidden, boxrule = 0sp,
    borderline west = {2pt}{0pt}{mytheoremfr},
    sharp corners, detach title,
    before upper = \tcbtitle\par\smallskip,
    coltitle = mytheoremfr, fonttitle = \bfseries\sffamily,
    description font = \mdseries, separator sign none,
    segmentation style={solid, mytheoremfr},
}{th}

% --- Corollary ---
\newtcbtheorem[number within=chapter]{corollary}{Corollary}{
    enhanced, breakable,
    colback = myp!10, frame hidden, boxrule = 0sp,
    borderline west = {2pt}{0pt}{myp!85!black},
    sharp corners, detach title,
    before upper = \tcbtitle\par\smallskip,
    coltitle = myp!85!black, fonttitle = \bfseries\sffamily,
    description font = \mdseries, separator sign none,
    segmentation style={solid, myp!85!black}
}{th}

% --- Lemma ---
\newtcbtheorem[number within=chapter]{lemma}{Lemma}{
    enhanced, breakable,
    colback = mylemmabg, frame hidden, boxrule = 0sp,
    borderline west = {2pt}{0pt}{mylemmafr},
    sharp corners, detach title,
    before upper = \tcbtitle\par\smallskip,
    coltitle = mylemmafr, fonttitle = \bfseries\sffamily,
    description font = \mdseries, separator sign none,
    segmentation style={solid, mylemmafr},
}{th}

% --- Proposition ---
\newtcbtheorem[number within=chapter]{prop}{Proposition}{
    enhanced, breakable,
    colback = mypropbg, frame hidden, boxrule = 0sp,
    borderline west = {2pt}{0pt}{mypropfr},
    sharp corners, detach title,
    before upper = \tcbtitle\par\smallskip,
    coltitle = mypropfr, fonttitle = \bfseries\sffamily,
    description font = \mdseries, separator sign none,
    segmentation style={solid, mypropfr},
}{th}

% --- Claim ---
\newtcbtheorem[number within=chapter]{claim}{Claim}{
    enhanced, breakable,
    colback = myg!10, frame hidden, boxrule = 0sp,
    borderline west = {2pt}{0pt}{myg},
    sharp corners, detach title,
    before upper = \tcbtitle\par\smallskip,
    coltitle = myg!85!black, fonttitle = \bfseries\sffamily,
    description font = \mdseries, separator sign none,
    segmentation style={solid, myg!85!black}
}{th}

% --- Exercise ---
\newtcbtheorem[number within=chapter]{exercise}{Exercise}{
    enhanced, breakable,
    colback = myexercisebg, frame hidden, boxrule = 0sp,
    borderline west = {2pt}{0pt}{myexercisefg},
    sharp corners, detach title,
    before upper = \tcbtitle\par\smallskip,
    coltitle = myexercisefg, fonttitle = \bfseries\sffamily,
    description font = \mdseries, separator sign none,
    segmentation style={solid, myexercisefg},
}{th}

% --- Example ---
\newtcbtheorem[number within=chapter]{example}{Example}{
    colback = myexamplebg, breakable,
    colframe = myexamplefr, coltitle = myexampleti,
    boxrule = 1pt, sharp corners, detach title,
    before upper=\tcbtitle\par\smallskip,
    fonttitle = \bfseries, description font = \mdseries,
    separator sign none, description delimiters parenthesis
}{ex}

% --- Definition ---
\newtcbtheorem[number within=chapter]{definition}{Definition}{
    enhanced, before skip=2mm, after skip=2mm, 
    colback=red!5, colframe=red!80!black, boxrule=0.5mm,
    attach boxed title to top left={xshift=1cm,yshift*=1mm-\tcboxedtitleheight}, 
    varwidth boxed title*=-3cm,
    boxed title style={frame code={
        \path[fill=tcbcolback]
        ([yshift=-1mm,xshift=-1mm]frame.north west)
        arc[start angle=0,end angle=180,radius=1mm]
        ([yshift=-1mm,xshift=1mm]frame.north east)
        arc[start angle=180,end angle=0,radius=1mm];
        \path[left color=tcbcolback!60!black,right color=tcbcolback!60!black,
        middle color=tcbcolback!80!black]
        ([xshift=-2mm]frame.north west) -- ([xshift=2mm]frame.north east)
        [rounded corners=1mm]-- ([xshift=1mm,yshift=-1mm]frame.north east)
        -- (frame.south east) -- (frame.south west)
        -- ([xshift=-1mm,yshift=-1mm]frame.north west)
        [sharp corners]-- cycle;
        },interior engine=empty,
    },
    fonttitle=\bfseries, title={#2},#1
}{def}

% --- Question ---
% --- Question ---
\newcounter{questioncounter}
\counterwithin{questioncounter}{chapter}
\newtcbtheorem[use counter=questioncounter]{question}{Question}{
    enhanced, breakable,
    colback=white, colframe=myb!80!black,
    attach boxed title to top left={yshift*=-\tcboxedtitleheight},
    fonttitle=\bfseries, title={#2},
    boxed title size=title,
    boxed title style={
        sharp corners, rounded corners=northwest,
        colback=tcbcolframe, boxrule=0pt,
    },
    underlay boxed title={
        \path[fill=tcbcolframe] (title.south west)--(title.south east)
        to[out=0, in=180] ([xshift=5mm]title.east)--
        (title.center-|frame.east)
        [rounded corners=3pt] |- (frame.north) -| cycle;
    },
    #1
}{def}

% --- Solution ---
\newtcolorbox{solution}{
    enhanced, breakable,
    colback=white, colframe=myg!80!black,
    attach boxed title to top left={yshift*=-\tcboxedtitleheight},
    title=Solution, boxed title size=title,
    boxed title style={
        sharp corners, rounded corners=northwest,
        colback=tcbcolframe, boxrule=0pt,
    },
    underlay boxed title={
        \path[fill=tcbcolframe] (title.south west)--(title.south east)
        to[out=0, in=180] ([xshift=5mm]title.east)--
        (title.center-|frame.east)
        [rounded corners=3pt] |- (frame.north) -| cycle;
    },
}

% --- Note ---
\newtcolorbox{note}[1][]{%
    enhanced jigsaw, colback=gray!20!white, colframe=gray!80!black,
    size=small, boxrule=1pt, title=\textbf{Note:-},
    halign title=flush center, coltitle=black, breakable,
    drop shadow=black!50!white,
    attach boxed title to top left={xshift=1cm,yshift=-\tcboxedtitleheight/2,yshifttext=-\tcboxedtitleheight/2},
    minipage boxed title=1.5cm,
    boxed title style={%
        colback=white, size=fbox, boxrule=1pt, boxsep=2pt,
        underlay={%
            \coordinate (dotA) at ($(interior.west) + (-0.5pt,0)$);
            \coordinate (dotB) at ($(interior.east) + (0.5pt,0)$);
            \begin{scope}
                \clip (interior.north west) rectangle ([xshift=3ex]interior.east);
                \filldraw [white, blur shadow={shadow opacity=60, shadow yshift=-.75ex}, rounded corners=2pt] (interior.north west) rectangle (interior.south east);
            \end{scope}
            \begin{scope}[gray!80!black]
                \fill (dotA) circle (2pt);
                \fill (dotB) circle (2pt);
            \end{scope}
        },
    },
    #1,
}

%==============================================================================
% CUSTOM COMMANDS / MACROS
%==============================================================================
\newcommand{\thm}[2]{\begin{theorem}{#1}{}#2\end{theorem}}
\newcommand{\cor}[2]{\begin{corollary}{#1}{}#2\end{corollary}}
\newcommand{\mlemma}[2]{\begin{lemma}{#1}{}#2\end{lemma}}
\newcommand{\mprop}[2]{\begin{prop}{#1}{}#2\end{prop}}
\newcommand{\clm}[3]{\begin{claim}{#1}{#2}#3\end{claim}}
\newcommand{\ex}[2]{\begin{example}{#1}{}#2\end{example}}
\newcommand{\dfn}[2]{\begin{definition}[colbacktitle=red!75!black]{#1}{}#2\end{definition}}
\newcommand{\qs}[2]{\begin{question}{#1}{}#2\end{question}}
\newcommand{\sol}{\begin{solution}} % Changed to simple environment start
\def\endsol{\end{solution}}
\newcommand{\nt}[1]{\begin{note}#1\end{note}}

% Common Math Shortcuts (Added since they were missing)
\newcommand{\bbR}{\mathbb{R}}
\newcommand{\bbC}{\mathbb{C}}
\newcommand{\RR}{\mathbb{R}} % Duplicate as used in text
\newcommand{\eps}{\epsilon}
\newcommand{\bs}[1]{\boldsymbol{#1}}

\newcommand*\circled[1]{\tikz[baseline=(char.base)]{\node[shape=circle,draw,inner sep=1pt] (char) {#1};}}

\newenvironment{myproof}[1][\proofname]{\proof[\bfseries #1: ]}{\endproof}

%==============================================================================
% TABLE OF CONTENTS STYLING
%==============================================================================
\contentsmargin{0cm}
\titlecontents{chapter}[4pc]
{\addvspace{30pt}%
    \begin{tikzpicture}[remember picture, overlay]%
        \draw[fill=doc!60,draw=doc!60] (-7,-.1) rectangle (-0.9,.5);%
        \pgftext[left,x=-3.7cm,y=0.2cm]{\color{white}\Large\bfseries Chapter\ \thecontentslabel}
    \end{tikzpicture}\color{doc!60}\large\bfseries}%
{}
{}
{\;\titlerule\;\large\bfseries Page \thecontentspage
    \begin{tikzpicture}[remember picture, overlay]
        \draw[fill=doc!60,draw=doc!60] (2pt,0) rectangle (4,0.1pt);
    \end{tikzpicture}}%

\titlecontents{section}[6.3pc]
{\addvspace{2pt}}
{\contentslabel[\thecontentslabel]{2pc}}
{}
{\hfill\small \thecontentspage}
[]

\titlecontents*{subsection}[8pc]
{\addvspace{1pt}\small}
{\contentslabel[\thecontentslabel]{2.5pc}}
{}
{}
% {\hfill\small \thecontentspage}  % Replace the curly brackets above with this line, if you want the subsections to have page numbers

\makeatletter
\renewcommand{\tableofcontents}{
    \chapter*{
      \vspace*{-20\p@}
      \begin{tikzpicture}[remember picture, overlay]%
          \pgftext[right,x=15cm,y=0.2cm]{\color{doc!60}\Huge\bfseries \contentsname}
          \draw[fill=doc!60,draw=doc!60] (13,-.75) rectangle (20,1);%
          \clip (13,-.75) rectangle (20,1);
          \pgftext[right,x=15cm,y=0.2cm]{\color{white}\Huge\bfseries \contentsname}
      \end{tikzpicture}}
    \@starttoc{toc}}
\makeatother
